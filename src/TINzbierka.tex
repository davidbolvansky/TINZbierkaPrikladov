\documentclass[]{article}
\usepackage[utf8]{inputenc}
\usepackage[slovak]{babel}
\usepackage{graphicx}
\usepackage{float}
\usepackage{mathtools}
\usepackage{amsfonts}
\usepackage{tikz}
\usepackage{hyperref}

\usetikzlibrary{arrows,shapes,automata,petri,positioning}

\hypersetup{
	colorlinks,
	citecolor=black,
	filecolor=black,
	linkcolor=black,
	urlcolor=black
}

\tikzset{
	place/.style={
		circle,
		thick,
		draw=black!75,
		fill=white!20,
		minimum size=6mm,
	},
	transition/.style={
		rectangle,
		thick,
		fill=black,
		minimum height=8mm,
		inner xsep=2pt
	}
}

\begin{document}
	
	
	\title{TIN - Študentská zbierka príkladov}
	\date{\today}
	
	\maketitle
	\newpage
	\tableofcontents
	\newpage
	
	\section{Chomského hierarchia}
	
	\begin{enumerate}
		\item 1. opravný termín skúšky 2017
		
		Formálne definujte pojem gramatika a pre každú triedu Chomského hierarchie uveďte typ gramatiky generujúcu jazyky tejto triedy.
		
		\item 2. opravný termín skúšky 2017
		
		Uvažujte Chomského hierarchiu jazykov rozšírenú o~triedu rekurzívnych jazykov a triedu deterministických bezkontextových jazykov. Pre každú triedu tejto klasifikácie uveďte a zdôvodnite, či je v~tejto triede rozhodnuteľný, alebo čiastočne rozhodnuteľný, problém náležitosti (členstva) daného reťazca do jazyka.
	\end{enumerate}

	\section{Regulárne jazyky}
	\begin{enumerate}
		\item 1. priebežný test 2018
		
		Pre deterministický konečný automat $A = (\{q_0,q_1,q_2,q_3\}, \{a,b,c\}, \delta, q_0, \{q_3\})$, kde $\delta$ je definovaná ako:
		
		$\delta(q_0, a) = q_1$ \hspace{10pt} $\delta(q_0, b) = q_0$ \hspace{10pt} $\delta(q_0, c) = q_0$
		
		$\delta(q_1, a) = q_2$ \hspace{10pt} $\delta(q_1, b) = q_0$ \hspace{10pt} $\delta(q_1, c) = q_0$
		
		$\delta(q_2, a) = q_2$ \hspace{10pt} $\delta(q_2, b) = q_3$ \hspace{10pt} $\delta(q_2, c) = q_0$
		
		$\delta(q_3, a) = q_3$ \hspace{10pt} $\delta(q_3, b) = q_3$ \hspace{10pt} $\delta(q_3, c) = q_3$
		
		zapíšte jazyk $L(A)$ v~tvare regulárneho výrazu. Ďalej zostrojte pravú lineárnu gramatiku $G$, pre ktorú platí, že $L(G) = L(A)$.
		
		\item 1. priebežný test 2018
		
		Uvážme nasledujúci problém $P$: pre nedeterministický konečný automat $A = (Q, \Sigma, \delta, q_0, F)$ rozhodnite, či je jazyk $L(A)$ nekonečný.
		
		\begin{itemize}
			\item Zapíšte stručne hlavnú myšlienku algoritmu, ktorý rieši problém $P$.
			\item Na základe prechodovej funkcie $\delta$ zapíšte formálne reláciu $R_\delta \subseteq Q \times Q$, ktorá popisuje, či je v~automate $A$ možný (priamy) medzi danou dvojicou stavov ($p,q$). Na základe tejto relácie a ich uzáveru zapíšte predikát, ktorý rozhoduje problém $P$.
			\item Demonštrujte použitie tohto predikátu na automatu $A = (\{q_0,q_1,q_2\}, \{a\}, \delta, q_0, \{q_2\})$, kde $\delta$ je definovaná ako:
			
			$\delta(q_0, a) = \{q_1, q_2\})$
			
			$\delta(q_1, a) = \{q_1, q_2\})$
			
			$\delta(q_2, a) = \emptyset)$
		\end{itemize}
		
		\item Riadny termín skúšky 2017
		
		Formálne definujte nedeterministický konečný automat, jeho konfiguráciu, reláciu prechodu medzi konfiguráciami a jazyk prijímaný týmto automatom.
		
		\item Riadny termín skúšky 2017
		
		Rozhodnite a dokážte, či jazyk
		
		$L = \{w \in \{a,b,c\}^* \mid \#_a(w)$ mod $2 = \#_b(w)$ mod $2\}$
		
		kde $\#_x(w)$ označuje počet znakov $x$ v~reťazci $w$ a $mod$ značí operáciu modulo, je regulárny.
		
		Poznámka: Pri dokazovaní, že je jazyk regulárny, stačí uviesť odpovedajúcu gramatiku alebo automat. Pri dokazovaní, že jazyk nie je regulárny, použite Pumping Lemma.
		
		\item Riadny termín skúšky 2017
		
		Formálne zapíšte obecný tvar sústavy rovníc nad regulárnymi výrazmi v~štandardnom tvare. Ďalej uvažujte jazyk generovaný gramatikou $G = (\{X,Y\}, \{x,y\}, P, X)$, kde $P$ je tvorená pravidlami:
		
		$X \rightarrow xyX \mid xxY \mid \varepsilon$
		
		$Y \rightarrow yY \mid x$
		
		Zostavením príslušnej sústavy rovníc nad regulárnymi výrazmi vo štandardnom tvare a jej riešením vyjadrite jazyk $L(G)$.
		
		Poznámka: Preferované riešenie neprevádza $G$ na ekvivalentný konečný automat.
		
		\item 1. opravný termín skúšky 2017
		
		Rozhodnite a dokážte, či jazyk
		
		$L = \{w \in \{a,b,c\}^* \mid \#_a(w) + \#_b(w) =  \#_c(w) \lor \#_c(w) \geq 2\}$
		
		kde $\#_x(w)$ označuje počet znakov $x$ v~reťazci $w$, je regulárny.
		
		Poznámka: Pri dokazovaní, že je jazyk regulárny, stačí uviesť odpovedajúcu gramatiku alebo automat. Pri dokazovaní, že jazyk nie je regulárny, použite Pumping Lemma.
		
		\item 2. opravný termín skúšky 2017
		
		Rozhodnite a dokážte, či nasledujúci jazyk je regulárny.
		
		$L = \{w \in \{a,b,c\}^* \mid \#_a(w) = \#_b(w) \land \#_b(w) \leq 2\}$
		
		Poznámka: $\#_x(w)$ označuje počet znakov $x$ v~reťazci $w$. Pri dokazovaní, že je jazyk regulárny, stačí uviesť odpovedajúcu gramatiku alebo automat. Pri dokazovaní, že jazyk nie je regulárny, použite Pumping Lemma.
		
		\item Riadny termín skúšky 2018
		
		Formálne definujte gramatiku typu 3, reláciu priamej derivácie a jazyk generovaný touto gramatikou.
		
		\item Riadny termín skúšky 2018
		
		Rozhodnite a dokážte, či jazyk
		
		$L = \{w \in \{a,b\}^* \mid \#_a(w)$ mod $2 < \#_a(w)$ mod $3\}$
		
		kde $\#_x(w)$ označuje počet znakov $x$ v~reťazci $w$ a $mod$ značí operáciu modulo, je regulárny.
		
		Poznámka: Pri dokazovaní, že je jazyk regulárny, stačí uviesť odpovedajúcu gramatiku alebo automat. Pri dokazovaní, že jazyk nie je regulárny, použite Pumping Lemma.
		
		\item 1. opravný termín skúšky 2018
		
		Formálne definujte redukovaný DKA, reláciu nerozlišiteľnosti. Zostrojte redukovaný DKA pre
		
		$L = \{w \in \{a,b,c\}^* \mid w$ obsahuje podreťazec $aab\}$
		
		\item 1. opravný termín skúšky 2018
			
		Rozhodnite a dokážte, či
		
		\begin{itemize}
			\item Existuje regulárny jazyk, ktorý nie je konečný ani co-konečný.
			\item Problém neprázdnosti KA je rozhodnuteľný.
			\item Trieda regulárnych jazykov je uzavretá na nekonečné zjednotenie, tj. pre každú nekonečnú množinu $\{L_0, L_1, L_2, \ldots\}$ regulárnych jazykov platí, že aj ich zjednotenie $L = \bigcup L_i$ je regulárny jazyk.
		\end{itemize}
	
	   \item 1. opravný termín skúšky 2018
	   
	   Dokážte, že nasledujúci jazyk nie je regulárny:
	    
	   $L = \{w \in \{a,b\}^* \mid \#a(w) \neq \#b(w)\}$ 
	   
	   Poznámka: nedoporučuje se použitie Pumping Lemma.
		
		
	\end{enumerate}

	\section{Bezkontextové jazyky}
	
	\begin{enumerate}
		\item 1. priebežný test 2018
		
		Pre bezkontextový jazyk
		
		$L = \{a^nb^mc^{3n} \mid n > 0 \land m$ je nepárne (liché)$\}$
		
		zostrojte a formálne zapíšte (v~zhode s~definíciou):
		\begin{itemize}
			\item bezkontextovú gramatiku $G$ takú, že $L(G) = L$
			\item zásobníkový automat $A$ taký, že $L(A) = L$
		\end{itemize}
		
		\item 1. priebežný test 2018
		
		Presne a formálne definujte gramatiky typu 0 a typu 2. Nech $G_1 = (N_1, \Sigma_1, P_1, S_1)$ a $G_2 = (N_2, \Sigma_2, P_2, S_2)$ sú gramatiky typu 2 a $N_1 \cap N_2 = \emptyset$. Zostrojte gramatiky $G_{\cdot}, G_*, G_{\cup}$ typu 2 také, že:
		
		$L(G_{\cdot}) = L(G_1) \cdot L(G_2)$
		
		$L(G_{*}) = L(G_1)^*$
		
		$L(G_{\cup}) = L(G_1) \cup L(G_2)$
		
		\item 2. priebežný test 2018
		
		Formálne zapíšte Pumping lemma pre bezkontextové jazyky.
		
		\item 2. priebežný test 2018
		
		Rozhodnite a dokážte, či jazyk $L$ nad abecedou $\Sigma = \{a,b,c\}$ je bezkontextový:
		
		$L = \{c^iw \mid i > 0 \land \#_a(w) \leq 3 * \#_b(w)\} \cap \{c^iww \mid i \geq 0 \land w \in \{a,b\}^*\}$
		
		kde $\#_x(w)$ označuje počet znakov $x$ v~reťazci $w$.
		
		\item 2. priebežný test 2018
		
		Pre deterministický zásobníkový automat (DZA) $M = (Q, \Sigma, \Gamma, \delta, q_0, Z_0, F)$ formálne definujte tvar prechodovej funkcie $\delta$ a konfiguráciu automatu $M$.
		
		\item 2. priebežný test 2018
		
		Nech $M = (Q, \Sigma, \Gamma, \delta, q_0, Z_0, F)$ je DZA. Dokážte, že jazyk
		
		$L = \{w \in \Sigma^* \mid w \in L(M) \land w$ obsahuje podreťazec $ab\}$
		
		je deterministický bezkontextový jazyk (je možné sa odkázať na vlastnosti bezkontextových jazykov z~prednášky).
		
		\item Riadny termín skúšky 2017
		
		Ukážte, že pre jazyk
		
		$L = \{wcw^R \mid w \in \{a,b\}^*\}$
		
		kde $w^R$ označuje reverzáciu reťazca $w$, platí Pumping Lemma pre bezkontextové jazyky pre hodnotu $k$ = 3 ($k$ je konštanta z~Pumping Lemma).
		
		\item Riadny termín skúšky 2017
		
		Navrhnite bezkontextovú gramatiku pre jazyk
		
		$L = \{a^nb^mc^md^n \mid n,m \geq 0\}$
		
		\item Riadny termín skúšky 2017
		
		Formálne definujte bezkontextovú gramatiku, priamu deriváciu, reláciu derivácie a jazyk generovaný touto gramatikou.
		
		\item 1 opravný termín skúšky 2017
		
		Formálne definujte (nedeterministický) zásobníkový automat, jeho konfiguráciu, reláciu prechodu medzi konfiguráciami a jazyk prijímaný týmto automatom.
		
		\item 1 opravný termín skúšky 2017
		
		Rozhodnite a dokážte, či nasledujúce jazyky nad abecedou $\Sigma = \{a,b,c\}$ sú bezkontextové:
		
		$L_1 = \{w \in \Sigma^* \mid (w = zcz^R \land z \in \{a,b\}^*) \lor \#_a(w) = \#_b(w)\}$
		
		$L_2 = \{w \in \Sigma^* \mid (w = zcz^R \land z \in \{a,b\}^*) \land \#_a(w) = \#_b(w)\}$
		
		kde $\#_x(w)$ označuje počet znakov $x$ v~reťazci $w$ a $w^R$ označuje reverzáciu reťazca $w$.
		
		\item 2 opravný termín skúšky 2017
		
		Uvažujme gramatiku $G = (\{S,A,B\}, \{a,b\}, P, S)$ s~pravidlami $P$:
		
		$S \rightarrow aSB \mid ASB \mid aa$
		
		$A \rightarrow aAa \mid B$
		
		$B \rightarrow bb \mid A$
		
		Zostroje (systematickým postupom z~prednášky) a formálne zapíšte zásobníkový automat $M$ taký, že $L(G) = L(M)$, ktorý modeluje syntaktickú analýzu zhora nadol.
		
		Zapíšte postupnosť konfigurácii stroje $M$ pre vstupný reťazec $bbaab$.
		
		\item 2 opravný termín skúšky 2017
		
		Rozhodnite a dokážte, či je nasledujúci jazyk bezkontextový.
		
		$L = \{w \in \{a,b,c\}^* \mid \#_a(w) + \#_c(w) \leq  \#_b(w) +  \#_c(w)\}$
		
		Poznámka: Pri dokazovaní, že je jazyk bezkontextový, stačí uviesť odpovedajúcu gramatiku alebo automat. Pri dokazovaní, že jazyk nie je bezkontextový, použite Pumping Lemma.
		
		\item Riadny termín skúšky 2018
		
		Formálne definujte nedeterministický zásobníkový automat, jeho konfiguráciu, reláciu prechodu a jazyk prijímaný týmto automatom.
		
		\item Riadny termín skúšky 2018
		
		Navrhnite bezkontextovú gramatiku pre jazyk
		
		$L = \{a^ib^jc^k \mid i,j,k > 0 \land (i \geq 3j \lor 2i \leq k)\}$
		
		\item Riadny termín skúšky 2018
		
		Ukážte, že pre jazyk
		
		$L = \{w \in \{a,b\}^* \mid \#_a(w) = \#_b(w)\}$
		
		kde $\#_x(w)$ označuje počet znakov $x$ v~reťazci $w$, platí Pumping Lemma pre bezkontextové jazyky pre hodnotu $k$ = 2 ($k$ je konštanta z~Pumping Lemma).
		
		\item Riadny termín skúšky 2018 
		
		Nech $M = (Q, \Sigma, \Gamma, \delta, q_0, Z_0, F)$ je nedeterministický zásobníkový automat. Popíšte konštrukciu nedeterministického zásobníkového automatu $M'$, pre ktorý platí:
		
		$L(M') = \{w \in \Sigma^* \mid w \in L(M) \land \#_a(w)$ mod $3 \neq 0\}$
		
		\item 1. opravný termín skúšky 2018
		
		Formálne definujte Pumping Lemma pre bezkontextové jazyky. Uveďte hlavné kroky dôkazu Pumping Lemma pre bezkontextové jazyky.
		
		\item 1. opravný termín skúšky 2018
		
		Formálne definujte reláciu prechodu u~DZA. Zostrojte DZA, ktorý akceptuje jazyk:
		
		$L = \{w \in \{a,b,c\}^* \mid \forall u \in \{a,b,c\}^*$ platí, že ak je $u$ prefix $w$, potom $\#a(u) \geq \#b(u)\}$
		
	\end{enumerate}

	\section{Algoritmy}
	
	\begin{enumerate}
		\item Riadny termín skúšky 2016
		
		Definujte sústavu rovníc nad regulárnymi výrazmi v~štandardnom tvare. Ďalej uvažujte obecnú lineárnu gramatiku $G$. Popíšte formálne algoritmus nájdenia regulárneho výrazu $R$ takého, že $L(G) = L(R)$, bez toho, aby bolo potrebné ku gramatike $G$ vytvárať ekvivalentný konečný automat a/alebo gramatiku $G$ transformovať. Algoritmus nájdenia regulárneho výrazu ilustrujte na príklade netriviálnej (s~rekurziou, aspoň 2 nonterminály a 4 pravidla) pravej lineárnej gramatiky $G$, ktorá nie je regulárna.
		
		\item Riadny termín skúšky 2017
		
		Zapíšte algoritmus (vrátane výpočtu množiny neterminálov $N_t = \{ A \mid A \xRightarrow{}^+ \varepsilon\}$), ktorý danú bezkontextovú gramatiku transformuje na jazykovo ekvivalentnú bezkontextovú gramatiku bez epsilon pravidiel.
		
		\item 1. opravný termín skúšky 2017
		
		Formálne definujte pojem $\varepsilon$-uzáver stavu RKA (rozšíreného konečného automatu, tj. nedeterministického automate s~$\varepsilon$ prechodmi) a formálne zapíšte algoritmus, ktorý v~polynomiálnom čase prevedie vstupný RKA na nedeterministický konečný automat bez $\varepsilon$ prechodov (NKA). Ďalej uvažujte nasledujúci RKA $A$:
		
		\begin{center}
			\begin{tikzpicture}[scale=0.2]
			\tikzstyle{every node}+=[inner sep=0pt]
			\draw [black] (18.3,-32.8) circle (3);
			\draw (18.3,-32.8) node {$p$};
			\draw [black] (29.7,-32.8) circle (3);
			\draw (29.7,-32.8) node {$q$};
			\draw [black] (39.9,-32.8) circle (3);
			\draw (39.9,-32.8) node {$s$};
			\draw [black] (39.9,-32.8) circle (2.4);
			\draw [black] (29.7,-45.3) circle (3);
			\draw (29.7,-45.3) node {$r$};
			\draw [black] (11.2,-32.8) -- (15.3,-32.8);
			\fill [black] (15.3,-32.8) -- (14.5,-32.3) -- (14.5,-33.3);
			\draw [black] (21.3,-32.8) -- (26.7,-32.8);
			\fill [black] (26.7,-32.8) -- (25.9,-32.3) -- (25.9,-33.3);
			\draw (24,-32.3) node [above] {$a$};
			\draw [black] (37.491,-34.547) arc (-66.68367:-113.31633:6.799);
			\fill [black] (37.49,-34.55) -- (36.56,-34.4) -- (36.95,-35.32);
			\draw (34.8,-35.6) node [below] {$b$};
			\draw [black] (31.6,-42.98) -- (38,-35.12);
			\fill [black] (38,-35.12) -- (37.11,-35.43) -- (37.88,-36.06);
			\draw (35.36,-40.48) node [right] {$a$};
			\draw [black] (20.32,-35.02) -- (27.68,-43.08);
			\fill [black] (27.68,-43.08) -- (27.51,-42.16) -- (26.77,-42.83);
			\draw (23.46,-40.51) node [left] {$\varepsilon$};
			\draw [black] (32.196,-31.175) arc (111.15248:68.84752:7.216);
			\fill [black] (37.4,-31.18) -- (36.84,-30.42) -- (36.48,-31.35);
			\draw (34.8,-30.19) node [above] {$\varepsilon$};
			\end{tikzpicture}
		\end{center}
		
		Pomocou zapísaného algoritmu preveďte $A$ na jazykovo ekvivalentný NKA (t.j. bez $\varepsilon$ prechodov).
		
		\item Riadny termín skúšky 2018
		
		Zapíšte algoritmus, ktorý daný nedeterministický konečný automat bez $\varepsilon$ prechodov prevedie na jazykovo ekvivalentný konečný automat. Algoritmus demonštrujte na automatu uvedenom nižšie.
		
		\begin{center}
			\begin{tikzpicture}[scale=0.2]
			\tikzstyle{every node}+=[inner sep=0pt]
			\draw [black] (18.3,-32.8) circle (3);
			\draw (18.3,-32.8) node {$1$};
			\draw [black] (28.6,-32.8) circle (3);
			\draw (28.6,-32.8) node {$2$};
			\draw [black] (28.6,-32.8) circle (2.4);
			\draw [black] (38.4,-32.8) circle (3);
			\draw (38.4,-32.8) node {$3$};
			\draw [black] (38.4,-32.8) circle (2.4);
			\draw [black] (11.2,-32.8) -- (15.3,-32.8);
			\fill [black] (15.3,-32.8) -- (14.5,-32.3) -- (14.5,-33.3);
			\draw [black] (16.977,-30.12) arc (234:-54:2.25);
			\draw (18.3,-25.55) node [above] {$a,b$};
			\fill [black] (19.62,-30.12) -- (20.5,-29.77) -- (19.69,-29.18);
			\draw [black] (21.3,-32.8) -- (25.6,-32.8);
			\fill [black] (25.6,-32.8) -- (24.8,-32.3) -- (24.8,-33.3);
			\draw (23.45,-32.3) node [above] {$a$};
			\draw [black] (31.6,-32.8) -- (35.4,-32.8);
			\fill [black] (35.4,-32.8) -- (34.6,-32.3) -- (34.6,-33.3);
			\draw (33.5,-32.3) node [above] {$b$};
			\draw [black] (37.077,-30.12) arc (234:-54:2.25);
			\draw (38.4,-25.55) node [above] {$a,b$};
			\fill [black] (39.72,-30.12) -- (40.6,-29.77) -- (39.79,-29.18);
			\end{tikzpicture}
		\end{center}
	
		\item 1. opravný termín skúšky 2018
		
		Uveďte hlavné kroky algoritmu, ktorý pre danú bezkontextovú gramatiku $G$ rozhoduje, či je jazyk $L(G)$ nekonečný. Algoritmus demonštrujte na bezkontextovej gramatike $G = (\{S,A\}, \{a,b\}, \{S \rightarrow aA, S \rightarrow Sb, A \rightarrow b\}, S)$.
		
	\end{enumerate}
	
	\section{Uzáverové vlastnosti}
	\begin{enumerate}
		\item 2. priebežný test 2018
	
		Rozhodnite a dokážte, či platia nasledujúce tvrdenia ($\mathcal{L}_2$ značí triedu všetkých bezkontextových jazykov a $\mathcal{L}_3$ značí triedu regulárnych jazykov):
		
		\begin{itemize}
			\item $\exists L_1 \in \mathcal{L}_2: \forall L_2 \in \mathcal{L}_2: L_1 \cap L_2 \in \mathcal{L}_2$
			
			\item $\exists L_1 \in \mathcal{L}_2 \setminus \mathcal{L}_3: \forall L_2 \in \mathcal{L}_3: L_1 \cap L_2 \in \mathcal{L}_3$
			
			\item Trieda bezkontextových jazykov nad abecedou $\Sigma = \{a,b\}$ je uzavrená vzhľadom k~binárnej operácii $\circ$ definovanej nasledovne:
			
			$L_1 \circ L_2 = \{w \in \Sigma^* \mid (w \in L_1 \land w \in L_2) \lor \vert w \vert > 1\}$
		\end{itemize}

		\item Riadny termín skúšky 2017
	
		Rozhodnite a dokážte, či pre jazyky nad abecedou $\Sigma$ platí:
		
		$\forall L_1 \in \mathcal{L}_3: \exists L_2 \in \mathcal{L}_2 \setminus \mathcal{L}_3: L_1 \cap L_2 \in \mathcal{L}_3$
			
		kde $\mathcal{L}_3$ a $\mathcal{L}_2$ značia triedu regulárnych resp. bezkontextových jazykov.
	
		\item 1. opravný termín skúšky 2017
	
		Rozhodnite a dokážte, či pre jazyky nad abecedou $\Sigma = \{a,b,c\}$ platí:
		
		\begin{itemize}
			\item $\forall L \in \mathcal{L}_3: \vert L \vert = \infty \rightarrow \Diamond L \in  \mathcal{L}_2$
			\item $\forall L \in \mathcal{L}_3: \vert L \vert = \infty \rightarrow \Diamond L \in  \mathcal{L}_2 \setminus \mathcal{L}_3$
		\end{itemize}	
	
		kde $\Diamond L = \{w \in L \mid \#_a(w) + \#_b(w) = \#_c(w)\}$

		\item 1. opravný termín skúšky 2017
	
		Rozhodnite a dokážte, či pre jazyky nad abecedou $\Sigma$ platí:
		
		$\forall L_1 \in \mathcal{L}_3: \exists L_2 \in \mathcal{L}_2 \setminus \mathcal{L}_3: L_1 \cup L_2 \in \mathcal{L}_3$
		
		$\mathcal{L}_3$ a $\mathcal{L}_2$ značia triedu regulárnych resp. bezkontextových jazykov.
		
		\item 2. opravný termín skúšky 2017
		
		Nech $\mathcal{L}_{DBJ}$ značí triedu deterministických bezkontextových jazykov a $\mathcal{L}_{3}$ triedu regulárnych jazykov. Rozhodnite a dokážte, či platí:
		
		$\exists L_1 \in \mathcal{L}_{DBJ}: \exists L_2 \in \mathcal{L}_{3}: L_1 \cap L_2 \notin \mathcal{L}_{DBJ}$
	
		\item Riadny termín skúšky 2018
		
		Nech $\mathcal{L}_{CK}$ značí triedu co-konečných jazykov, ktorých komplement je konečný. Rozhodnite a dokážte, či platí:
		
		$\forall L_1, L_2 \in \mathcal{L}_{CK}$ je jazyk $L_1 \cdot L_2$ regulárny
	\end{enumerate}

	\section{Turingove stroje}
	
	\begin{enumerate}
		\item 2. priebežný test 2017
		
		Definujte prechodovú funkciu NTS, reťazec prijímaný TS, jazyk prijímaný TS. TS zadaný prechodovou funkciou má na vstupe $\Delta$abca$\Delta^{w}$. Doplňte 4 pravidlá tak, aby výstup bol $\Delta$acba$\Delta^w$.
		
		\item 2. priebežný test 2018
		
		Pre deterministický Turingov stroj $M = (Q, \Sigma, \Gamma, \delta, q_0, q_f)$ formálne definujte tvar prechodovej funkcie $\delta$, konfiguráciu stroja $M$ a reláciu prechodu $\vdash_M$ medzi konfiguráciami.
		
		\item 2. priebežný test 2018
		
		Zostrojte a formálne zapíšte deterministický Turingov stroj $M$ o~najviac 4 stavoch a 4 prechodoch tak, aby platilo $(q_0, \Delta a^i\Delta^w, 0) {\vdash^*}_M (q_f, \Delta b^i\Delta^w, n)$, kde i,n $\geq$ 0.
	\end{enumerate}

	
	\section{Diagonalizácia}
	
	\begin{enumerate}
		\item 1. opravný termín skúšky 2017
		
		Pomocou techniky diagonalizácie dokážte, že existuje jazyk, ktorý nie je rekurzívne vyčísliteľný.
		
		\item Riadny termín skúšky 2016
		
		Dokážte, že existuje totálna funkcia $f: \mathbb{N} \rightarrow \mathbb{N}$, ktorá nie je primitívne rekurzívna.
		
		\item 1. opravný termín skúšky 2016
		
		Diagonalizáciou dokážte, že existuje jazyk, ktorý nie je kontextový, ale je rekurzívny.
		
	\end{enumerate}
	
	\section{Redukcie, rekurzívne a rekurzívne vyčísliteľné jazyky}
	\begin{enumerate}
		\item Riadny termín skúšky 2017
		
		Formálne definujte pojem redukcie jazyka $L_1$ na jazyk $L_2$ a zapíšte príslušné tvrdenia (implikácie) pre určovanie rozhodnuteľnosti resp. nerozhodnuteľnosti jazykov.
		
		\item Riadny termín skúšky 2017
		
		Rozhodnete a dokážte, či sú rekurzívne vyčísliteľné jazyky uzavreného vzhľadom k~operácii pozitívna iterácia +.
		
		\item Riadny termín skúšky 2017
		
		Rozhodnite a dokážte, či existuje rekurzívne vyčísliteľný jazyk $L_1$ a rekurzívny jazyk $L_2$, pre ktoré platí $L_2 \leq L_2$ (tj. $L_1$ sa redukuje na $L_2$).
		
		\item 1. opravný termín skúšky 2017
		
		Rozhodnite a dokážte, či existuje jazyk $L$, ktorý nie je rekurzívny, ale je rekurzívne vyčísliteľný, a jeho doplnok $\overline{L}$ je tiež rekurzívne vyčísliteľný.
		
		\item Riadny termín skúšky 2016
		
		Rozhodnite a dokážte, či jazyk:
		
		$L_1 = \{\langle M \rangle \mid \exists w \in \Sigma^*$ také, že $M$ zastaví na $w\}$ je rekurzívny
		
		$L_2 = \{\langle M \rangle \mid \exists w \in \Sigma^*$ také, že $M$ nezastaví na $w$ behom 17 krokov$\}$ je rekurzívne vyčísliteľný
	
		$\langle M \rangle$ označuje kód Turingovho stroja so vstupnou abecedou $\Sigma$
		
		\item 1. opravný termín skúšky 2016
		
		Rozhodnite a dokážte, či jazyk:
		
		$L_1 = \{\langle M \rangle \mid \forall w \in \Sigma^*$ také, že $M$ nezastaví na $w\}$ je rekurzívny
		
		$L_2 = \{\langle M \rangle \mid \forall w \in \Sigma^*$ také, že $M$ zastaví na $w$ behom 17 krokov$\}$ je rekurzívne vyčísliteľný
		
		$\langle M \rangle$ označuje kód Turingovho stroja so vstupnou abecedou $\Sigma$
		
		\item 1. opravný termín skúšky 2016
		
		Uveďte jazyk, ktorý je rekurzívne vyčísliteľný, ale nie je rekurzívny, a jeho komplement je tiež rekurzívne vyčísliteľný.
		
		\item Riadny termín skúšky 2017
		
		Pre abecedu $\Sigma = \{a, b\}$ rozhodnite a dokážte, či:
		
		$L_1 = \{\langle M \rangle \mid M$ je Turingov stroj taký, že $\vert L(M) \cap \{a,b\} \vert = 1\}$ je rekurzívny
		
		$L_2 = \{\langle M \rangle \mid M$ je Turingov stroj taký, že $\vert L(M) \cap \{a,b\} \vert \geq 1\}$ je rekurzívne vyčísliteľný
		
		$L_3 = \{\langle M \rangle \mid M$ je Turingov stroj, pre ktorý platí, že existuje $w \in \{a,b\}^{42}$ také, že $M$ zastaví na $w$ do $\vert w \vert$ krokov$\}$ je rekurzívny
		
		Poznámka: $\langle M \rangle$ označuje kód Turingovho stroja $M$.
		
		\item 1. opravný termín skúšky 2017
		
		$L_1 = \{\langle M \rangle \mid M$ je Turingov stroj taký, že $L(M)$ je bezkontextový jazyk$\}$ je rekurzívne vyčísliteľný
		
		$L_2 = \{\langle M \rangle \mid M$ je Turingov stroj taký, že $\vert L(M) \vert \geq 3\}$ je rekurzívne vyčísliteľný
		
		Poznámka: $\langle M \rangle$ označuje kód Turingovho stroja $M$.
		
		\item Riadny termín skúšky 2018
		
		Definujte triedu rekurzívnych a rekurzívne vyčísliteľných jazykov. Ďalej pre každú triedu uveďte jazyk, ktorý do danej triedy patrí, a jazyk, ktorý do triedy nepatrí.
		
		\item Riadny termín skúšky 2018
		
		Definujte jazyk $L_{HP}$, ktorý špecifikuje problém zastavenia. Ďalej rozhodnite a dokážte, či existuje rekurzívny jazyk $L$, pre ktorý platí, že $L \leq L_{HP}$ (tj. $L$ sa redukuje na $L_{HP}$).
		
		\item Riadny termín skúšky 2018
		
		Pre abecedu $\Sigma = \{a, b\}$ rozhodnite a dokážte, či:
		
		$L_1 = \{\langle M \rangle \mid M$ je Turingov stroj taký, že $\vert L(M) \vert  > \vert \Sigma \vert\}$ je rekurzívny
		
		$L_2 = \{\langle M \rangle \mid M$ je Turingov stroj, pre ktorý platí, že $\exists w \in L(M) : \vert w \vert > \vert \langle M \rangle \vert\}$ je rekurzívne vyčísliteľný
		
		Poznámka: $\langle M \rangle$ označuje reťazec, ktorý kóduje Turingov stroj $M$. V~dôkazoch stačí uviesť hlavnú myšlienku redukcie či konštrukcie požadovaného Turingovho stroja.
	\end{enumerate}
	
	\section{Zložitosť}
	
	\begin{enumerate}
		\item Riadny termín skúšky 2016
		
		Rozhodnite a dokážte, či platí:
		
		$n^3 \notin \mathcal{O}(n^2)$
		
		$2n^n + 2n + n \in \mathcal{O}(n^2 - 2n - 2)$
		
		$\{a^nb^nc^n \mid n \geq 0\} \in DTIME[n]$
		
		\item Riadny termín skúšky 2017
		
		Definujte formálne časovú zložitosť Turingových strojov a triedu jazykov $DTIME[n^5]$.
		
		\item Riadny termín skúšky 2017
		
		Rozhodnite a dokážte, či platí:
		
		$n^3 \in \mathcal{O}(10n^2 + 100)$
		
		$10n^2 + 100 \in \mathcal{O}(n^3)$
		
		\item Riadny termín skúšky 2017
		
		Rozhodnite a dokážte, či platí:
		
		$L_1, L_2 \in DTIME[n^3] \xRightarrow[]{} \{ uv \mid u \in  L_1 \land v \in L_2\} \in NTIME[n^3]$
		
		\item 1. opravný termín skúšky 2017
		
		Definujte formálne:
		
		\begin{itemize}
			\item pre funkciu $f: \mathbb{N} \rightarrow \mathbb{N}$ množinu $\mathcal{O}(f(n))$
			\item priestorovú zložitosť nedeterministických Turingových strojov
		\end{itemize}
	
		\item 1. opravný termín termín skúšky 2017
		
		Rozhodnite a dokážte, či platí:
		
		$L \in DTIME[n^4] \xRightarrow[]{} \{ u_1, u_2 \ldots u_k \mid k \geq 1, \forall 1 \leq i \leq k: u_i \in L\} \in NTIME[n^4]$
		
		\item Riadny termín skúšky 2018
		
		Formálne definujte:
		
		\begin{itemize}
			\item priestorovú zložitosť nedeterministických Turingových strojov, ktorý prijíma jazyk $L$
			\item asymptotické horné obmedzenie funkcie $f: \mathbb{N} \rightarrow \mathbb{N}$ (tj. $\mathcal{O}(f(n))$
			\item triedu jazykov $NSPACE[2^n]$
		\end{itemize}
	
		\item Riadny termín skúšky 2018
		
		Pre $\Sigma = \{a,b,c\}$ rozhodnite a dokážte, či platí:
		
		$L \in DTIME[n^5] \xRightarrow[]{} \{ w \in \Sigma^* \mid \exists w' \in L$ také, že $w'$ je podslovo slova $w\} \in DTIME[n^7]$
	\end{enumerate}

	\section{NP problémy, polynomiálna redukcia}

	\begin{enumerate}
		\item Riadny termín skúšky 2017
		
		Definujte formálne, kedy je jazyk NP-úplný a dokážte, že nasledujúci jazyk je NP-úplný:
		
		$L = \{(\phi_1, \phi_2) \mid \phi_1, \phi_2$ sú výrokové formule v~konjunktívnej normálnej forme, pre ktoré existujú dve rôzne valuácie premenných $v_1$ a $v_2$ také, že $\phi_1(v_1) \neq \phi_2(v_1) \land \phi_1(v_2) \neq \phi_2(v_2)\}$
		
		Poznámka: $\phi_i(v_i) \in \{true, false\}$ označuje, či je formula $\phi_i$ pravdivá pri valuácií premenných $v_i$.
		
		\item 1. opravný termín skúšky 2017
		
		Definujte formálne, kedy je jazyk NP-úplný a dokážte, že nasledujúci jazyk je NP-úplný:
		
		$L = \{(\phi, n) \mid \phi$ je výroková formula nad premennými $x_1, \ldots, x_k$ v~konjunktívnej normálnej forme, $n \in \mathbb{N_0}$ a naviac platí, že existuje valuácia $v$ premenných $x_1, \ldots, x_k$, ktorá splňuje $\phi$, a pre ktorú platí $E(v) \geq n\}$,
		
		kde $E(v) \in \mathbb{N}$ značí číslo, ktorého binárny zápis $n_1, \ldots, n_k$ je definovaný nasledovným spôsobom:
		
		$n_i =
		\left\{
		\begin{array}{ll}
			0  & \mbox{ak } v(x_i) = true \\
			1 & \mbox{ak } v(x_i) = false
		\end{array}
		\right.$
		
		\item Riadny termín skúšky 2018
		
		Definujte formálne, kedy je jazyk NP-úplný. Ďalej uveďte hlavnú myšlienku dôkazu, že jazyk $L$ definovaný nižšie je NP-úplný:
		
		$L = \{(\phi_1, \phi_2) \mid \phi_1, \phi_2$ sú výrokové formule v~konjunktívnej normálnej forme, pre ktoré existuje valuácia premenných $\vec{v}$ taká, že $\phi_1(\vec{v}) \neq \phi_2(\vec{v})\}$
		
		Poznámka: $\phi_i(\vec{v}) \in \{true, false\}$ označuje, či je formula $\phi_i$ pravdivá pri valuácií premenných $\vec{v}$.
	\end{enumerate}

	\section{Vyčíslitelné funkcie}
	
	\begin{enumerate}
		\item 1. opravný termín skúšky 2017
		
		Pomocou počiatočných funkcií a operátorov kombinácie, kompozície a primitívnej rekurzie vyjadrite funkciu:
		
		$tplus(x,y) = x + 3y$
		
		Nepoužívajte žiadne ďalšie funkcie zavedené na prednáškach mimo počiatočných funkcií. Nepoužívajte zjednodušenú syntax zápisu funkcií \-- dodržujte presne definičný tvar operátorov kombinácie, kompozície a primitívne rekurzie.
		
		\item 2. opravný termín skúšky 2017
		
		Pomocou počiatočných funkcií a operátorov kombinácie, kompozície a primitívnej rekurzie vyjadrite funkciu:
		
		$tplus(x,y) = 3x + 2y$
		
		Nepoužívajte žiadne ďalšie funkcie zavedené na prednáškach mimo počiatočných funkcií. Nepoužívajte zjednodušenú syntax zápisu funkcií \-- dodržujte presne definičný tvar operátorov kombinácie, kompozície a primitívne rekurzie.
	\end{enumerate}

	\section{Petriho siete}
	
	\begin{enumerate}
		\item Riadny termín skúšky 2017
		
		Definujte formálne P/T Petriho siete. V~zhode s~touto definíciou popíšte sieť na obrázku (všetky miesta majú neobmedzenú kapacitu). Ďalej popíšte množinu výpočtových postupností tejto Petriho siete ako jazyk nad množinou jej prechodov.
		
		\begin{tikzpicture}[node distance=1cm,>=stealth',,bend angle=45,auto]
		\node [place,tokens=1,label=below:$P_1$] (p1) {};
		\node [transition,label=above:$T_1$] (t1) [above= of p1] {}
		edge[pre]   (p1);
		\node [transition,label=below:$T_2$] (t2) [right= of p1] {}
		edge[pre]   (p1);
		\node [transition] (t2x) [above= of p1] {}
		edge[post]   (p1);
		\node [place,label=below:$P_3$] (p3) [right= of t2] {};
		\node [transition] (t2) [right= of p1] {}
		edge[post]   (p3);
		\node [place,label=above:$P_2$] (p2) [right= of t1] {};
		\node [transition] (t1) [above= of p1] {}
		edge[post]   (p2);
		\node [transition,label=above:$T_3$] (t3) [above= of p3] {}
		edge[pre]   (p2);
		\node [transition] (t3) [above= of p3] {}
		edge[pre]   (p3);
		\node [transition] (t3x) [above= of p3] {}
		edge[post]   (p3);
		\end{tikzpicture}
		
		\item 1. opravný termín skúšky 2017
		
		Pre P/T Petriho sieť $N = (P,T,F,W,K,M_0)$ definujte formálne:
		
		\begin{itemize}
			\item predpis pre výpočet nasledujúceho značenia $M'$ zo značenia $M$ pri prevediteľnom prechode $t$ (tj. platí $M[t\rangle M'$)
			\item množinu $[M_0\rangle$ dosiahnuteľných značení siete $N$
			\item (obecnú) prechodovú funkciu $\delta: [M_0\rangle \times T^* \rightarrow [M_0\rangle$
		\end{itemize}
	
		\item Riadny termín skúšky 2018
		
		Definujte formálne P/T Petriho siete. V~zhode s~touto definíciou popíšte sieť na obrázku (všetky miesta majú neobmedzenú kapacitu). Ďalej zapíšte prevediteľnú postupnosť prechodov a odpovedajúcich značení, v~ktorej sa vyskytujú všetky prechody (použite zavedenú notáciu $M[t\rangle M'$).
		
		\begin{tikzpicture}[node distance=1cm,>=stealth',bend angle=45,auto]
		\node [place,tokens=1,label=below:$P_1$] (p1) {};
		\node [transition,label=below:$T_1$] (t1) [right= of p1] {}
		edge[pre]   (p1);
		\node [transition] (t1x) [right= of p1] {}
		edge[post]   (p1);
		\node [place,label=right:$P_2$] (p2) [above right= of t1]{};
		\node [transition] (t1) [right= of p1] {}
		edge[post, bend right]   (p2);
		\node [transition,label=above:$T_2$] (t2) [above left= of p2] {}
		edge[pre, bend left] node {3} (p2);
		\node [transition] (t2) [above left= of p2] {}
		edge[pre, bend left] node {3} (p2);
		\node [transition] (t2) [above left= of p2] {}
		edge[pre, bend right] (p1);
		\node [place,tokens=3,label=above:$P_3$] (p3) [left= of t2]{};
		\node [transition] (t2x) [right= of p3] {}
		edge[post]  node[swap] {3} (p3);
		\node [transition,label=left:$T_3$] (t3) [above left= of p1] {}
		edge[pre, bend left] node {6} (p3);
		\node [transition] (t3x) [above left= of p1] {}
		edge[post, bend right] (p1);
		\end{tikzpicture}
		
	\end{enumerate}

\end{document}