\documentclass[]{article}
\usepackage[utf8]{inputenc}
\usepackage[slovak]{babel}
\usepackage{graphicx}
\usepackage{float}
\usepackage{mathtools}
\usepackage{amsfonts}
\usepackage{tikz}

\newcommand\ImageItem[2][]{%
	\hfill\raisebox{\dimexpr-\height+\baselineskip}{\includegraphics[#1]{#2}}\hfill~}

\begin{document}
	
	
	\title{TIN - Zbierka príkladov}
	\date{\today}
	
	\maketitle
	\newpage
	\tableofcontents
	\newpage
	
	\section{Chomského hierarchia}
	
	\begin{figure}[H]
		\includegraphics[width=\textwidth]{tasks/chomsky/task1.png}
	\end{figure}
	
	\begin{figure}[H]
		\includegraphics[width=\textwidth]{tasks/chomsky/task2.png}
	\end{figure}

	\section{Regulárne jazyky}
	
	\begin{figure}[H]
		\includegraphics[width=\textwidth]{tasks/regularne/task1.png}
	\end{figure}
	
	\begin{figure}[H]
		\includegraphics[width=\textwidth]{tasks/regularne/task2.png}
	\end{figure}
	
	\begin{figure}[H]
		\includegraphics[width=\textwidth]{tasks/regularne/task3.png}
	\end{figure}
	
	\begin{figure}[H]
		\includegraphics[width=\textwidth]{tasks/regularne/task4.png}
	\end{figure}
	
	\begin{figure}[H]
		\includegraphics[width=\textwidth]{tasks/regularne/task5.png}
	\end{figure}
	
	\begin{figure}[H]
		\includegraphics[width=\textwidth]{tasks/regularne/task6.png}
	\end{figure}
	
	\begin{figure}[H]
		\includegraphics[width=\textwidth]{tasks/regularne/task7.png}
	\end{figure}
	
	\begin{figure}[H]
		\includegraphics[width=\textwidth]{tasks/regularne/task8.png}
	\end{figure}
	
	\begin{figure}[H]
		\includegraphics[width=\textwidth]{tasks/regularne/task9.png}
	\end{figure}

	\section{Bezkontextové jazyky}
	
	\begin{figure}[H]
		\includegraphics[width=\textwidth]{tasks/bezkontextove/task1.png}
	\end{figure}

	\begin{figure}[H]
		\includegraphics[width=\textwidth]{tasks/bezkontextove/task2.png}
	\end{figure}
	
	\begin{figure}[H]
		\includegraphics[width=\textwidth]{tasks/bezkontextove/task3.png}
	\end{figure}
	
	\begin{figure}[H]
		\includegraphics[width=\textwidth]{tasks/bezkontextove/task4.png}
	\end{figure}
	
	\begin{figure}[H]
		\includegraphics[width=\textwidth]{tasks/bezkontextove/task5.png}
	\end{figure}
	
	\begin{figure}[H]
		\includegraphics[width=\textwidth]{tasks/bezkontextove/task6.png}
	\end{figure}
	
	\begin{figure}[H]
		\includegraphics[width=\textwidth]{tasks/bezkontextove/task7.png}
	\end{figure}
	
	\begin{figure}[H]
		\includegraphics[width=\textwidth]{tasks/bezkontextove/task8.png}
	\end{figure}

	\begin{figure}[H]
		\includegraphics[width=\textwidth]{tasks/bezkontextove/task9.png}
	\end{figure}
	
	\begin{figure}[H]
		\includegraphics[width=\textwidth]{tasks/bezkontextove/task10.png}
	\end{figure}
	
	\begin{figure}[H]
		\includegraphics[width=\textwidth]{tasks/bezkontextove/task11.png}
	\end{figure}
	
	\begin{figure}[H]
		\includegraphics[width=\textwidth]{tasks/bezkontextove/task12.png}
	\end{figure}
	
	\begin{figure}[H]
		\includegraphics[width=\textwidth]{tasks/bezkontextove/task13.png}
	\end{figure}
	
	\begin{figure}[H]
		\includegraphics[width=\textwidth]{tasks/bezkontextove/task14.png}
	\end{figure}

	\section{Algoritmy}
	
	\begin{enumerate}
		\item Riadny termín skúšky 2016
		
		Definujte sústavu rovníc nad regulárnymi výrazmi v štandardnom tvare. Ďalej uvažujte obecnú lineárnu gramatiku $G$. Popíšte formálne algoritmus nájdenia regulárneho výrazu $R$ takého, že $L(G) = L(R)$, bez toho, aby bolo potrebné ku gramatike $G$ vytvárať ekvivalentný konečný automat a/alebo gramatiku $G$ transformovať. Algoritmus nájdenia regulárneho výrazu ilustrujte na príklade netriviálnej (s rekurziou, aspoň 2 nonterminály a 4 pravidla) pravej lineárnej gramatiky $G$, ktorá nie je regulárna.
		
		\item Riadny termín skúšky 2017
		
		Zapíšte algoritmus (vrátane výpočtu množiny neterminálov $N_t = \{ A \mid A \xRightarrow{}^+ \varepsilon\}$), ktorý danú bezkontextovú gramatiku transformuje na jazykovo ekvivalentnú bezkontextovú gramatiku bez epsilon pravidiel.
		
		\item 1. opravný termín skúšky 2017
		
		Formálne definujte pojem $\varepsilon$-uzáver stavu RKA (rozšíreného konečného automatu, tj. nedeterministického automatu s $\varepsilon$ pravidlami) a formálne zapíšte algoritmus, ktorý v polynomiálnom čase prevedie vstupný RKA na nedeterministický konečný automat bez $\varepsilon$ prechodov (NKA). Ďalej uvažujte nasledujúci RKA $A$:
		
		\begin{center}
			\begin{tikzpicture}[scale=0.2]
			\tikzstyle{every node}+=[inner sep=0pt]
			\draw [black] (18.3,-32.8) circle (3);
			\draw (18.3,-32.8) node {$p$};
			\draw [black] (28.9,-32.8) circle (3);
			\draw (28.9,-32.8) node {$q$};
			\draw [black] (39.9,-32.8) circle (3);
			\draw (39.9,-32.8) node {$s$};
			\draw [black] (39.9,-32.8) circle (2.4);
			\draw [black] (29.7,-45.3) circle (3);
			\draw (29.7,-45.3) node {$r$};
			\draw [black] (11.2,-32.8) -- (15.3,-32.8);
			\fill [black] (15.3,-32.8) -- (14.5,-32.3) -- (14.5,-33.3);
			\draw [black] (21.3,-32.8) -- (25.9,-32.8);
			\fill [black] (25.9,-32.8) -- (25.1,-32.3) -- (25.1,-33.3);
			\draw (23.6,-32.3) node [above] {$a$};
			\draw [black] (37.437,-34.48) arc (-66.83172:-113.16828:7.72);
			\fill [black] (37.44,-34.48) -- (36.51,-34.34) -- (36.9,-35.25);
			\draw (34.4,-35.6) node [below] {$b$};
			\draw [black] (31.132,-30.831) arc (118.76404:61.23596:6.792);
			\fill [black] (31.13,-30.83) -- (32.07,-30.88) -- (31.59,-30.01);
			\draw (34.4,-29.49) node [above] {$\varepsilon$};
			\draw [black] (31.6,-42.98) -- (38,-35.12);
			\fill [black] (38,-35.12) -- (37.11,-35.43) -- (37.88,-36.06);
			\draw (35.36,-40.48) node [right] {$\varepsilon$};
			\draw [black] (20.32,-35.02) -- (27.68,-43.08);
			\fill [black] (27.68,-43.08) -- (27.51,-42.16) -- (26.77,-42.83);
			\draw (23.46,-40.51) node [left] {$\varepsilon$};
			\end{tikzpicture}
		\end{center}
		
		Pomocou zapísaného algoritmu preveďte $A$ na jazykovo ekvivalentný NKA (t.j. bez $\varepsilon$ prechodov).
		
		\item 2. opravný termín skúšky 2017, Riadny termín skúšky 2018
		
		Zapíšte algoritmus, ktorý daný nedeterministický konečný automat bez $\varepsilon$ prechodov prevedie na jazykovo ekvivalentný konečný automat. Algoritmus demonštrujte na automatu uvedenom nižšie.
		
		\begin{center}
			\begin{tikzpicture}[scale=0.2]
			\tikzstyle{every node}+=[inner sep=0pt]
			\draw [black] (18.3,-32.8) circle (3);
			\draw (18.3,-32.8) node {$1$};
			\draw [black] (28.6,-32.8) circle (3);
			\draw (28.6,-32.8) node {$2$};
			\draw [black] (28.6,-32.8) circle (2.4);
			\draw [black] (38.4,-32.8) circle (3);
			\draw (38.4,-32.8) node {$3$};
			\draw [black] (38.4,-32.8) circle (2.4);
			\draw [black] (11.2,-32.8) -- (15.3,-32.8);
			\fill [black] (15.3,-32.8) -- (14.5,-32.3) -- (14.5,-33.3);
			\draw [black] (16.977,-30.12) arc (234:-54:2.25);
			\draw (18.3,-25.55) node [above] {$a,b$};
			\fill [black] (19.62,-30.12) -- (20.5,-29.77) -- (19.69,-29.18);
			\draw [black] (21.3,-32.8) -- (25.6,-32.8);
			\fill [black] (25.6,-32.8) -- (24.8,-32.3) -- (24.8,-33.3);
			\draw (23.45,-32.3) node [above] {$a$};
			\draw [black] (31.6,-32.8) -- (35.4,-32.8);
			\fill [black] (35.4,-32.8) -- (34.6,-32.3) -- (34.6,-33.3);
			\draw (33.5,-32.3) node [above] {$b$};
			\draw [black] (37.077,-30.12) arc (234:-54:2.25);
			\draw (38.4,-25.55) node [above] {$a,b$};
			\fill [black] (39.72,-30.12) -- (40.6,-29.77) -- (39.79,-29.18);
			\end{tikzpicture}
		\end{center}
	\end{enumerate}
	
	\section{Uzáverové vlastnosti}
	
	Definicia, co je to uzavretost triedy jazykov. Dokaz neuzavretosti deterministickych bezkontextovych jazykov na operacie prienik a zjednotenie. Zadana binarna relacia, dokazat, ze je uzavrena. $L1 \circ L2 = \{uv \mid u \in L1 \land v \in L2 \land \vert uv \vert \leq 5\}$.
	
	\begin{figure}[H]
		\includegraphics[width=\textwidth]{tasks/vlastnosti/task1.png}
	\end{figure}

	\begin{figure}[H]
		\includegraphics[width=\textwidth]{tasks/vlastnosti/task2.png}
	\end{figure}
	
	\begin{figure}[H]
		\includegraphics[width=\textwidth]{tasks/vlastnosti/task3.png}
	\end{figure}


	\begin{figure}[H]
		\includegraphics[width=\textwidth]{tasks/vlastnosti/task4.png}
	\end{figure}
	
	\begin{figure}[H]
		\includegraphics[width=\textwidth]{tasks/vlastnosti/task5.png}
	\end{figure}

	\begin{figure}[H]
		\includegraphics[width=\textwidth]{tasks/vlastnosti/task6.png}
	\end{figure}

	\section{Turingove stroje}
	
	\begin{enumerate}
		\item 2. priebežný test 2017
		
		Definujte prechodovú funkciu NTS, reťazec prijímaný TS, jazyk prijímaný TS. TS zadaný prechodovou funkciou má na vstupu $\Delta$abca$\Delta^{w}$. Doplňte 4 pravidlá tak, aby výstup bol $\Delta$acba$\Delta^w$.
		
		\item 2. priebežný test 2018
		
		Pre deterministický Turingov stroj $M = (Q, \Sigma, \Gamma, \sigma, q_0, q_f)$ formálne definujte tvar prechodovej funkcie $\sigma$, konfiguráciu stroja $M$ a reláciu prechodu $\vdash_M$ medzi konfiguráciami.
		
		\item 2. priebežný test 2018
		
		Zostrojte a formálne zapíšte deterministický Turingov stroj $M$ o najviac 4 stavoch a 4 prechodoch tak, aby platilo $(q_0, \Delta a^i\Delta^w, 0) {\vdash^*}_M (q_f, \Delta b^i\Delta^w, n)$, kde i,n $\geq$ 0.
	\end{enumerate}

	
	\section{Diagonalizácia}
	
	\begin{enumerate}
		\item 1. opravný termín skúšky 2017
		
		Pomocou techniky diagonalizácie dokážte, že existuje jazyk, ktorý nie je rekurzívne vyčísliteľný.
		\item Riadny termín skúšky 2016
		
		Dokážte, že existuje totálna funkcia $f: \mathbb{N} \rightarrow \mathbb{N}$, ktorá nie je primitívne rekurzívna.
		
	\end{enumerate}
	
	\section{Redukcie, rekurzívne a rekurzívne vyčísliteľné jazyky}
	\begin{enumerate}
		\item Riadny termín skúšky 2017
		
		Formálne definujte pojem redukcie jazyka $L_1$ na jazyk $L_2$ a zapíšte príslušné tvrdenia (implikácie) pre určovanie rozhodnuteľnosti resp. nerozhodnuteľnosti jazykov.
		
		\item Riadny termín skúšky 2017
		
		Rozhodnete a dokážte, či sú rekurzívne vyčísliteľné jazyky uzavreného vzhľadom k operácii pozitívna iterácia +.
		
		\item Riadny termín skúšky 2017
		
		Rozhodnite a dokážte, či existuje rekurzívne vyčísliteľný jazyk $L_1$ a rekurzívny jazyk $L_2$, pre ktoré platí $L_2 \leq L_2$ (tj. $L_1$ sa redukuje na $L_2$).
		
		\item 1. opravný termín skúšky 2017
		
		Rozhodnite a dokážte, či existuje jazyk $L$, ktorý nie je rekurzívny, ale je rekurzívne vyčísliteľný, a jeho doplnok $\overline{L}$ je tiež rekurzívne vyčísliteľný.
	\end{enumerate}
	
	\begin{figure}[H]
		\includegraphics[width=\textwidth]{tasks/redukcie/task4.png}
	\end{figure}

	\begin{figure}[H]
		\includegraphics[width=\textwidth]{tasks/redukcie/task6.png}
	\end{figure}
	
	\begin{figure}[H]
		\includegraphics[width=\textwidth]{tasks/redukcie/task8.png}
	\end{figure}
	
	\section{Zložitosť}
	
	\begin{enumerate}
		\item Riadny termín skúšky 2017
		
		Definujte formálne časovú zložitosť Turingových strojov a triedu jazykov $DTIME[n^5]$.
		
		\item Riadny termín skúšky 2017
		
		Rozhodnite a dokážte, či platí:
		
		$n^3 \in \mathcal{O}(10n^2 + 100)$
		
		$10n^2 + 100 \in \mathcal{O}(n^3)$
		
		\item Riadny termín skúšky 2017
		
		Rozhodnite a dokážte, či platí:
		
		$L_1, L_2 \in DTIME[n^3] \xRightarrow[]{} \{ uv \mid u \in  L_1 \land v \in L_2\} \in NTIME[n^3]$
		
		\item 1. opravný termín skúšky 2017
		
		Definujte formálne:
		
		\begin{itemize}
			\item pre funkciu $f: \mathbb{N} \rightarrow \mathbb{N}$ množinu $\mathcal{O}(f(n))$
			\item priestorovú zložitosť nedeterministických Turingových strojov
		\end{itemize}
	
		\item 1. opravný termín termín skúšky 2017
		
		Rozhodnite a dokážte, či platí:
		
		$L \in DTIME[n^4] \xRightarrow[]{} \{ u_1, u_2 \ldots u_k \mid k \geq 1, \forall 1 \leq i \leq k: u_i \in L\} \in NTIME[n^4]$
		
		\item Riadny termín skúšky 2018
		
		Formálne definujte:
		
		\begin{itemize}
			\item priestorovú zložitosť nedeterministických Turingových strojov, ktorý prijíma jazyk $L$
			\item asymptotické horné obmedzenie funkcie $f: \mathbb{N} \rightarrow \mathbb{N}$ (tj. $\mathcal{O}(f(n))$
			\item triedu jazykov $NSPACE[2^n]$
		\end{itemize}
	\end{enumerate}

	\section{NP problémy, polynomiálna redukcia}
	
	\begin{figure}[H]
		\includegraphics[width=\textwidth]{tasks/np/task1.png}
	\end{figure}

	\begin{figure}[H]
		\includegraphics[width=\textwidth]{tasks/np/task2.png}
	\end{figure}
	
	\begin{figure}[H]
		\includegraphics[width=\textwidth]{tasks/np/task3.png}
	\end{figure}

	\section{Vyčíslitelné funkcie}

	\begin{figure}[H]
		\includegraphics[width=\textwidth]{tasks/funkcie/task1.png}
	\end{figure}

	\section{Petriho siete}
	
	\begin{figure}[H]
		\includegraphics[width=\textwidth]{tasks/petri/task1.png}
	\end{figure}

	\begin{figure}[H]
		\includegraphics[width=\textwidth]{tasks/petri/task2.png}
	\end{figure}

	\begin{figure}[H]
		\includegraphics[width=\textwidth]{tasks/petri/task3.png}
	\end{figure}

	
\end{document}